%%%%%%%%%%%%%%%%%
% This is an example CV created using altacv.cls (v1.1.5, 1 December 2018) written by
% LianTze Lim (liantze@gmail.com), based on the
% Cv created by BusinessInsider at http://www.businessinsider.my/a-sample-resume-for-marissa-mayer-2016-7/?r=US&IR=T
%
%% It may be distributed and/or modified under the
%% conditions of the LaTeX Project Public License, either version 1.3
%% of this license or (at your option) any later version.
%% The latest version of this license is in
%%    http://www.latex-project.org/lppl.txt
%% and version 1.3 or later is part of all distributions of LaTeX
%% version 2003/12/01 or later.
%%%%%%%%%%%%%%%%

%% If you are using \orcid or academicons
%% icons, make sure you have the academicons
%% option here, and compile with XeLaTeX
%% or LuaLaTeX.
% \documentclass[10pt,a4paper,academicons]{altacv}

%% Use the "normalphoto" option if you want a normal photo instead of cropped to a circle
% \documentclass[10pt,a4paper,normalphoto]{altacv}

\documentclass[10pt,a4paper,ragged2e]{altacv}

%% AltaCV uses the fontawesome and academicon fonts
%% and packages.
%% See texdoc.net/pkg/fontawecome and http://texdoc.net/pkg/academicons for full list of symbols. You MUST compile with XeLaTeX or LuaLaTeX if you want to use academicons.

% Change the page layout if you need to
\geometry{left=1cm,right=9cm,marginparwidth=6.8cm,marginparsep=1.2cm,top=1.25cm,bottom=1.25cm}

% Change the font if you want to, depending on whether
% you're using pdflatex or xelatex/lualatex
\ifxetexorluatex
  % If using xelatex or lualatex:
  \setmainfont{Carlito}
\else
  % If using pdflatex:
  \usepackage[utf8]{inputenc}
  \usepackage[T1]{fontenc}
  \usepackage[default]{lato}
\fi

\usepackage[spanish]{babel}
\usepackage{multicol}
\usepackage{nicefrac}

% Change the colours if you want to
\definecolor{VividPurple}{HTML}{3E0097}
\definecolor{SlateGrey}{HTML}{2E2E2E}
\definecolor{LightGrey}{HTML}{666666}
\colorlet{heading}{VividPurple}
\colorlet{accent}{VividPurple}
\colorlet{emphasis}{SlateGrey}
\colorlet{body}{LightGrey}

% Change the bullets for itemize and rating marker
% for \cvskill if you want to
\renewcommand{\itemmarker}{{\small\textbullet}}
\renewcommand{\ratingmarker}{\faCircle}

%% sample.bib contains your publications
% \addbibresource{sample.bib}
\addbibresource{CV.bib}

\begin{document}
\name{Javier Alonso Silva}
\tagline{Ingeniero de Computadores, desarrollador \textit{software} y administrador de sistemas}
% Cropped to square from https://en.wikipedia.org/wiki/Marissa_Mayer#/media/File:Marissa_Mayer_May_2014_(cropped).jpg, CC-BY 2.0
\photo{2.5cm}{me}
\personalinfo{
    \begin{tabular}{ l l l }
    \email{contact@javinator9889.com} & \telegram{javinator9889} & \location{Madrid, España} \\
    \linkedin{linkedin.com/in/javinator9889} & \homepage{javinator9889.com} & \twitter{javinator9889} \\
    \end{tabular}
}

%% Make the header extend all the way to the right, if you want.
\begin{fullwidth}
\makecvheader
\end{fullwidth}

%% Depending on your tastes, you may want to make fonts of itemize environments slightly smaller
\AtBeginEnvironment{itemize}{\small}

%% Provide the file name containing the sidebar contents as an optional parameter to \cvsection.
%% You can always just use \marginpar{...} if you do
%% not need to align the top of the contents to any
%% \cvsection title in the "main" bar.
\cvsection[page1sidebar]{Experiencia}

\cvevent{Ingeniero Embebido I+D}{Teldat}{Septiembre 2021 -- \textit{actualidad}}{Tres Cantos, España}
\begin{itemize}
  \item Desarrollo de un sistema operativo basado en Linux para \textit{routers}.
  \item Programación en tiempo real y \textit{hardware} de dispositivos de comunicación.
  \item Tareas de desarrollo multidisciplinar con el equipo de Alemania.
\end{itemize}
\divider

\cvevent{Profesor Grado Superior}{Retamar FP}{Enero 2021 -- Junio 2021}{Pozuelo de Alarcón, España}
\begin{itemize}
  \item Profesor de la asignatura ISO (\textit{ASIR1}).
  \item Profesor de la asignatura PSP (\textit{DAM2}).
  \item Mentorización de proyectos y prácticas en empresas.
\end{itemize}
\divider

\cvevent{Desarrollador \textit{full--stack}}{\textit{Freelance}}{Dic. 2016 -- \textit{actualidad}}{}
Desarrollar mis propias aplicaciones y soluciones para empresas.
\begin{itemize}
  \item Proyecto ADAF, para \href{https://saturnolabs.com/}{SaturnoLabs}.
  \item Otros proyectos privados.
\end{itemize}

\cvsection{Educación}
\cvevent{Máster en Software de Sistemas Distribuidos y Empotrados}{Universidad Politécnica de Madrid}{Sept. 2020 -- 2021}{Campus Sur, Madrid}
\begin{itemize}
  \item CGPA: $\nicefrac{3.7}{4}$ (junio, 2021)
  \item Matrículas de Honor: \textit{Ingeniería de Software de Sistemas}, \textit{Programación Avanzada de Smartphones}, \textit{Seguridad en Sistemas y Redes}, \textit{Big Data y ML}
\end{itemize}

\divider

\cvevent{Ingeniería de Computadores}{Universidad Politécnica de Madrid}{Sept. 2016 -- Nov. 2020}{Campus Sur, Madrid}
\begin{itemize}
    \item CGPA: $\nicefrac{2.95}{4}$ (noviembre, 2020).
    \item Matrícula de honor: \textit{Programación Orientada a Objetos}, \textit{Trabajo Fin de Grado}.
    \item Asistente de laboratorio (\textit{Tecnología de Computadores} / 2019).
\end{itemize}

\divider

\cvevent{Actividades diversas y talleres}{}{}{}
\begin{itemize}
  \item Curso \href{https://factoriatalento.es/}{Factoría de Talento} (2019).
  \item \href{https://www.meetup.com/es-ES/quantummadrid/}{Quantum Madrid} (desde 2019).
  \item Curso \textit{Clean Code} (\href{https://www.autentia.com/}{Autentia}, 2017).
  \item Coordinador de voluntarios -- \href{https://www.upm.es/e-politecnica/?p=11246}{campaña de donación de médula ósea} (2019, 2020).
  \item Voluntario \href{https://eventos.upm.es/12427/detail/european-conference-on-software-architectures-2018-ecsa18.html}{ECSA} (2018).
  \item Participante activo \href{http://madrid.pyladies.com/}{PyLadies} (desde 2017).
\end{itemize}

\clearpage

\cvsection[page2sidebar]{Mis trabajos}

\nocite{*}

{\large \printinfo{\faCode}{\textbf{Aplicaciones}}}

\begin{center}
    \appMaterialCard[8cm]{Handwashing reminder}{Kotlin}{Una aplicación que te recuerda cuándo lavarte las manos.}{Javinator9889/Handwashing-reminder}{https://github.com/Javinator9889/Handwashing-reminder}
    
    \appMaterialCard[8cm]{ThdKernel}{C}{Kernel de Linux personalizado para ASUS Vivobook Pro 15 (N580GD).}{Javinator9889/ThdKernel}{https://github.com/Javinator9889/ThdKernel}
    
    \appMaterialCard[8cm]{YouTubeMDBot}{Python}{Bot de \textit{Telegram} para descargar vídeos desde YouTube, centrado en la velocidad y eficiencia.}{Javinator9889/YouTubeMDBot}{https://github.com/Javinator9889/YouTubeMDBot}

    \appMaterialCard[8cm]{Bonsai AIO}{C/C++/Arduino}{Controla el nivel de agua, temperatura, tiempo, luminosidad del bonsai y riego automático con una placa Arduino.}{Javinator9889/BonsaiAIO}{https://github.com/Javinator9889/BonsaiAIO}
        
    \appMaterialCard[8cm]{KernelUpgrader}{Python}{Solución para actualizar el kernel de Linux fácilmente.}{Javinator9889/KernelUpgrader}{https://github.com/Javinator9889/KernelUpgrader}

    \appMaterialCard[8cm]{Calculator}{Kotlin}{Una potente calculadora básica que está diseñada para funcionar lo más rápido posible ocupando poco espacio.}{Javinator9889/Calculator}{https://github.com/Javinator9889/Calculator}

    \appMaterialCard[8cm]{pArm -- S2}{C}{pArm software embebido en el dispositivo final.}{pArm-TFG/pArm-S2}{https://github.com/pArm-TFG/pArm-S2}
    
    % \appMaterialCard[8cm]{pyGle}{Python}{Una herramienta para buscar en toda la web usando Google.}{Javinator9889/pyGle}{https://github.com/Javinator9889/pyGle}
\end{center}

% \printbibliography[heading=pubtype,title={\printinfo{\faGroup}{Conference Proceedings}},type=inproceedings]

% If the NEXT page doesn't start with a \cvsection but you'd
% still like to add a sidebar, then use this command on THIS
% page to add it. The optional argument lets you pull up the
% sidebar a bit so that it looks aligned with the top of the
% main column.
% \addnextpagesidebar[-1ex]{page3sidebar}


\end{document}
