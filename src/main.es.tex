%%%%%%%%%%%%%%%%%
% This is an example CV created using altacv.cls (v1.1.5, 1 December 2018) written by
% LianTze Lim (liantze@gmail.com), based on the
% Cv created by BusinessInsider at http://www.businessinsider.my/a-sample-resume-for-marissa-mayer-2016-7/?r=US&IR=T
%
%% It may be distributed and/or modified under the
%% conditions of the LaTeX Project Public License, either version 1.3
%% of this license or (at your option) any later version.
%% The latest version of this license is in
%%    http://www.latex-project.org/lppl.txt
%% and version 1.3 or later is part of all distributions of LaTeX
%% version 2003/12/01 or later.
%%%%%%%%%%%%%%%%

%% If you are using \orcid or academicons
%% icons, make sure you have the academicons
%% option here, and compile with XeLaTeX
%% or LuaLaTeX.
% \documentclass[10pt,a4paper,academicons]{altacv}

%% Use the "normalphoto" option if you want a normal photo instead of cropped to a circle
% \documentclass[10pt,a4paper,normalphoto]{altacv}

\documentclass[10pt,a4paper,ragged2e]{altacv}

%% AltaCV uses the fontawesome and academicon fonts
%% and packages.
%% See texdoc.net/pkg/fontawecome and http://texdoc.net/pkg/academicons for full list of symbols. You MUST compile with XeLaTeX or LuaLaTeX if you want to use academicons.

% Change the page layout if you need to
\geometry{left=1cm,right=9cm,marginparwidth=6.8cm,marginparsep=1.2cm,top=1.25cm,bottom=1.25cm}

% Change the font if you want to, depending on whether
% you're using pdflatex or xelatex/lualatex
\ifxetexorluatex
  % If using xelatex or lualatex:
  \setmainfont{Carlito}
\else
  % If using pdflatex:
  \usepackage[utf8]{inputenc}
  \usepackage[T1]{fontenc}
  \usepackage[default]{lato}
\fi

\usepackage[spanish]{babel}
\usepackage{multicol}
\usepackage{nicefrac}

% Change the colours if you want to
\definecolor{VividPurple}{HTML}{3E0097}
\definecolor{SlateGrey}{HTML}{2E2E2E}
\definecolor{LightGrey}{HTML}{666666}
\colorlet{heading}{VividPurple}
\colorlet{accent}{VividPurple}
\colorlet{emphasis}{SlateGrey}
\colorlet{body}{LightGrey}

% Change the bullets for itemize and rating marker
% for \cvskill if you want to
\renewcommand{\itemmarker}{{\small\textbullet}}
\renewcommand{\ratingmarker}{\faCircle}

%% sample.bib contains your publications
% \addbibresource{sample.bib}
\addbibresource{CV.bib}

\begin{document}
\name{Javier Alonso Silva}
\tagline{\textsc{Ingeniero trabajando activamente en hacer del mundo un lugar mejor}}
% Cropped to square from https://en.wikipedia.org/wiki/Marissa_Mayer#/media/File:Marissa_Mayer_May_2014_(cropped).jpg, CC-BY 2.0
\photo{2.5cm}{static/me}
\personalinfo{
  \begin{tabularx}{\linewidth}{ l l }
    \email{contact@javinator9889.com}        & \location{Madrid, España} \\
    \linkedin{linkedin.com/in/javinator9889} & \linktree{javinator9889}  \\
  \end{tabularx}
}

%% Make the header extend all the way to the right, if you want.
\begin{fullwidth}
  \makecvheader
\end{fullwidth}

%% Depending on your tastes, you may want to make fonts of itemize environments slightly smaller
\AtBeginEnvironment{itemize}{\small}

%% Provide the file name containing the sidebar contents as an optional parameter to \cvsection.
%% You can always just use \marginpar{...} if you do
%% not need to align the top of the contents to any
%% \cvsection title in the "main" bar.
\cvsection[sidebar/1/sidebar.es]{Experiencia}

\cvevent{Ingeniero I+D}{\accentsub{https://teldat.com}{Teldat S.A.}}{Sept. 2021 -- \textit{actualidad}}{}
\begin{itemize}
  \item Desarrollo del NGFW que usan nuestros productos.
  \item \textit{Drivers} y \textit{networking} en el kernel de Linux.
  \item CI y orquestación de tests -- Orcha.
\end{itemize}
\divider

\cvevent{Miembro}{\accentsub{https://www.novatalent.com/}{Nova}}{Mayo 2023 -- \textit{actualidad}}{}
{\small Nova es una red de acceso según méritos en donde el top 3\% del talento conecta, desarrolla y
acelera su carrera profesional, unos con otros.}

\divider

\cvevent{Profesor Grado Superior}{\accentsub{https://retamar.com/}{Retamar FP}}{Enero 2021 -- Junio 2021}{}
\begin{itemize}
  \item Profesor de las asignaturas ISO (\textit{ASIR1}) y PSP (\textit{DAM2}).
  \item Mentorización de proyectos y prácticas en empresas.
\end{itemize}
\divider

\cvevent{Desarrollador \textit{full--stack}}{\textit{Freelance}}{Dic. 2016 -- \textit{actualidad}}{}
\begin{itemize}
  \item Proyecto ADAF, para \href{https://saturnolabs.com/}{SaturnoLabs}.
  \item Otros proyectos privados (diseño \textit{hardware}, Android apps, \dots).
\end{itemize}

\cvsection{Proyectos}

\cvevent{Ponente en la PyConUS}{\accentsub{https://www.python.org/psf-landing/}{The Python Software Foundation}}{2023}{Salt Lake City, Utah}

\begin{itemize}
  \item Charla ``Orcha: Procesamiento Masivo Paralelo (MPP) y diseño de APIs''.
  \item \textit{Lightning Talk} sobre qué es la PyConES.
\end{itemize}

\divider

\cvevent{Organización PyConES}{\accentsub{https://es.python.org/}{Python España}}{2022, 2023}{España}

\begin{itemize}
  \item Gestión de otros voluntarios, recursos e infraestructura de la conferencia.
  \item Contacto con empresas para opciones de patrocinio, gestión de recursos, plazos, etc.
\end{itemize}

\divider

\cvevent{Coordinador de Voluntarios -- Equipo Médula}{Universidad Politénica de Madrid}{2018}{Madrid, España}

\begin{itemize}
  \item Gestión de todos los voluntarios que participaron en el proyecto (+60).
  \item Formación y contacto con cientos de alumnos, explicándoles el proceso y cómo unirse.
\end{itemize}

% \divider

\clearpage

\cvsection[sidebar/2/sidebar.es]{Otra información}

\nocite{*}

{\large \printinfo{\faCode}{\textbf{Aplicaciones}}}

\begin{center}
  \appMaterialCard[8cm]{Handwashing reminder}{Kotlin}{Una aplicación que te recuerda cuándo lavarte las manos.}{Javinator9889/Handwashing-reminder}{https://github.com/Javinator9889/Handwashing-reminder}

  \appMaterialCard[8cm]{pArm -- S2}{C}{pArm software embebido en el dispositivo final.}{pArm-TFG/pArm-S2}{https://github.com/pArm-TFG/pArm-S2}

  \appMaterialCard[8cm]{Orcha}{Python}{Framework de orquestación y ejecución masiva de procesos.}{Javinator9889/orcha}{https://github.com/Javinator9889/orcha}

  % \appMaterialCard[8cm]{pyGle}{Python}{Una herramienta para buscar en toda la web usando Google.}{Javinator9889/pyGle}{https://github.com/Javinator9889/pyGle}
  \divider

  ¿Interesado en ver más proyectos? Échale un vistazo a mi GitHub:
  
  \href{https://github.com/Javinator9889/}{\githubsymbol{} Javinator9889}
\end{center}

\printbibliography[heading=pubtype,title={\printinfo{\faFileTextO}{\textbf{Artículos}}}, type=article]

% \printbibliography[heading=pubtype,title={\printinfo{\faGroup}{Conference Proceedings}},type=inproceedings]

% If the NEXT page doesn't start with a \cvsection but you'd
% still like to add a sidebar, then use this command on THIS
% page to add it. The optional argument lets you pull up the
% sidebar a bit so that it looks aligned with the top of the
% main column.
% \addnextpagesidebar[-1ex]{page3sidebar}


\end{document}
